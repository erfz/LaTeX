% !TEX TS-program = pdflatex
% !TEX encoding = UTF-8 Unicode

% This is a simple template for a LaTeX document using the "article" class.
% See "book", "report", "letter" for other types of document.

\documentclass[12pt]{article} % use larger type; default would be 10pt

\usepackage[utf8]{inputenc} % set input encoding (not needed with XeLaTeX)

%%% Examples of Article customizations
% These packages are optional, depending whether you want the features they provide.
% See the LaTeX Companion or other references for full information.

%%% PAGE DIMENSIONS
\usepackage{geometry} % to change the page dimensions
\geometry{a4paper} % or letterpaper (US) or a5paper or....
% \geometry{margin=2in} % for example, change the margins to 2 inches all round
% \geometry{landscape} % set up the page for landscape
%   read geometry.pdf for detailed page layout information

\usepackage{graphicx} % support the \includegraphics command and options

% \usepackage[parfill]{parskip} % Activate to begin paragraphs with an empty line rather than an indent

%%% PACKAGES
\usepackage{booktabs} % for much better looking tables
\usepackage{array} % for better arrays (eg matrices) in maths
\usepackage{paralist} % very flexible & customisable lists (eg. enumerate/itemize, etc.)
\usepackage{verbatim} % adds environment for commenting out blocks of text & for better verbatim
\usepackage{subfig} % make it possible to include more than one captioned figure/table in a single float
% These packages are all incorporated in the memoir class to one degree or another...

%%% HEADERS & FOOTERS
\usepackage{fancyhdr} % This should be set AFTER setting up the page geometry
\pagestyle{fancy} % options: empty , plain , fancy
\renewcommand{\headrulewidth}{0pt} % customise the layout...
\lhead{}\chead{}\rhead{}
\lfoot{}\cfoot{\thepage}\rfoot{}

%%% SECTION TITLE APPEARANCE
\usepackage{sectsty}
\allsectionsfont{\sffamily\mdseries\upshape} % (See the fntguide.pdf for font help)
% (This matches ConTeXt defaults)

%%% ToC (table of contents) APPEARANCE
\usepackage[nottoc,notlof,notlot]{tocbibind} % Put the bibliography in the ToC
\usepackage[titles,subfigure]{tocloft} % Alter the style of the Table of Contents
\renewcommand{\cftsecfont}{\rmfamily\mdseries\upshape}
\renewcommand{\cftsecpagefont}{\rmfamily\mdseries\upshape} % No bold!

\usepackage{mathtools}
\usepackage{gensymb}
\usepackage{dcolumn}
\newcolumntype{d}[1]{D{.}{\cdot}{#1}}

%%% END Article customizations

%%% The "real" document content comes below...

\title{Rod Hitting Ball Lab}
\author{Jason Guo}
%\date{} % Activate to display a given date or no date (if empty),
         % otherwise the current date is printed 

\begin{document}
\maketitle

\section{Data}

\begin{center}
    \begin{tabular}{ | l | d{4} | l | }
    \hline
    Quantity & \mathrm{Value} & Unit \\ \hline
    $L$ (rod length) & 0.381 & m \\ \hline
    $l_{cm}$ (distance from axis to rod/mass COM) & 0.299 & m \\ \hline
    $l_{top}$ (distance from axis to top of rod) & 0.025 & m \\ \hline
    $l_{weight}$ (distance from bottom of rod to weight COM) & 0.02 & m \\ \hline
    $H$ (final track height above ground) & 0.98 & m \\ \hline
    $h_i$ (initial track height) & 0.09 & m \\ \hline
    $h_f$ (final track height) & 0.124 & m \\ \hline
    $R$ (ball radius) & 0.0285 & m \\ \hline
    $M$ (ball mass) & 0.08155 & kg \\ \hline
    $m$ (total rod \& weight mass) & 0.1017 & kg \\ \hline
    $m_{rod}$ (rod mass) & 0.02654 & kg \\ \hline
    $m_{weight}$ (weight mass) & 0.07516 & kg \\ \hline
    $\theta_i$ (initial rod angle) & 60 & \degree \\ \hline
    \end{tabular}
\end{center}

\section{Calculations}

\subsection{MOI of rod and mass}

Let the MOI of the rod and mass $\coloneqq I_{rm}$. Let the distance from the axis to the weight's COM $\coloneqq d = L - l_{top} - \frac{l_{weight}}{2}$.
$$I_{rm} = \frac{1}{12}m_{rod}L^2 + m_{rod}(\frac{L}{2} - l_{top})^2 + m_{weight}d^2.$$
Thus, $I_{rm} \approx 0.01\, kgm^2$.

\subsection{Angular speed of rod on collision}

From energy conservation, we have
$$mgl_{cm}(1-cos\,\theta_i) = \frac{1}{2}I_{rm}\omega_i^2.$$
Thus $\omega_i \approx 5.46\, rad/s$.

\subsection{Angular momentum}

Taking counterclockwise to be positive, we have
$$L = I_{rm}\omega_i = I_{rm}\omega_f + Mvd - I_{b}\frac{v}{R}$$
where $I_b = \frac{2}{5}MR^2$ is the MOI of the ball about its COM, $v$ is the linear velocity of the ball's COM, and $d$ is the same as above.

We must make an assumption about the nature of the collision; namely, that it is perfectly inelastic.
Then, we have $v_{contact} = \omega_fd = v$, where $v_{tip}$ is the linear velocity of the rod's contact point with the ball. This contact point is a distance $d$ from the axis.

Substituting into the angular momentum expression, we get $v \approx 0.97\,m/s$ and $\omega_f \approx 2.8\,rad/s$.

\section{Results}

\subsection{Final rod angle}

We reuse the energy conservation expression and find that $\omega_f \approx 2.8\,rad/s$ corresponds to $\theta_f \approx 30\degree$.

\subsection{Ball Travel Distance}

To simplify calculations, the total kinetic energy $K$ of the ball is $K = K_{linear} + K_{rotation} = \frac{1}{2}Mv^2 + \frac{1}{2}I_b(\frac{v}{R})^2 = \frac{7}{10}Mv^2$ for an arbitrary ball COM linear velocity $v$.
By energy conservation,
$$K_i = K_f + Mg(h_f - h_i)$$
since the ball goes up a height of $h_f - h_i$ before leaving the track. Let $u$ denote the ball COM's linear velocity right before leaving the track. Then, $u \approx 0.68\,m/s$.
The travel time of the ball $t$ is given by
$$-H = -\frac{1}{2}gt^2 \implies t \approx 0.45\,s.$$
Thus, the ball will travel a distance $ut \approx 0.306\,m \approx 31\,cm$ off the table.

\end{document}
