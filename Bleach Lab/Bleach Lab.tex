% !TEX TS-program = pdflatex
% !TEX encoding = UTF-8 Unicode

% This is a simple template for a LaTeX document using the "article" class.
% See "book", "report", "letter" for other types of document.

\documentclass[11pt]{article} % use larger type; default would be 10pt

\usepackage[T1]{fontenc}
\usepackage[utf8]{inputenc} % set input encoding (not needed with XeLaTeX)

%%% Examples of Article customizations
% These packages are optional, depending whether you want the features they provide.
% See the LaTeX Companion or other references for full information.

%%% PAGE DIMENSIONS
\usepackage{geometry} % to change the page dimensions
\geometry{a4paper} % or letterpaper (US) or a5paper or....
% \geometry{margin=2in} % for example, change the margins to 2 inches all round
% \geometry{landscape} % set up the page for landscape
%   read geometry.pdf for detailed page layout information

\usepackage{graphicx} % support the \includegraphics command and options

% \usepackage[parfill]{parskip} % Activate to begin paragraphs with an empty line rather than an indent

%%% PACKAGES
\usepackage{booktabs} % for much better looking tables
\usepackage{array} % for better arrays (eg matrices) in maths
\usepackage{paralist} % very flexible & customisable lists (eg. enumerate/itemize, etc.)
\usepackage{verbatim} % adds environment for commenting out blocks of text & for better verbatim
\usepackage{subfig} % make it possible to include more than one captioned figure/table in a single float
% These packages are all incorporated in the memoir class to one degree or another...

%%% HEADERS & FOOTERS
\usepackage{fancyhdr} % This should be set AFTER setting up the page geometry
\pagestyle{fancy} % options: empty , plain , fancy
\renewcommand{\headrulewidth}{0pt} % customise the layout...
\lhead{}\chead{}\rhead{}
\lfoot{}\cfoot{\thepage}\rfoot{}

%%% SECTION TITLE APPEARANCE
\usepackage{sectsty}
\allsectionsfont{\sffamily\mdseries\upshape} % (See the fntguide.pdf for font help)
% (This matches ConTeXt defaults)

%%% ToC (table of contents) APPEARANCE
\usepackage[nottoc,notlof,notlot]{tocbibind} % Put the bibliography in the ToC
\usepackage[titles,subfigure]{tocloft} % Alter the style of the Table of Contents
\renewcommand{\cftsecfont}{\rmfamily\mdseries\upshape}
\renewcommand{\cftsecpagefont}{\rmfamily\mdseries\upshape} % No bold!

%%% END Article customizations

%%% The "real" document content comes below...

\title{Analysis of Commercial Bleach Lab}
\author{Jason Guo}
%\date{} % Activate to display a given date or no date (if empty),
         % otherwise the current date is printed 

\begin{document}
\maketitle

\section{Objective}

To standardize a solution of sodium thiosulfate using a potassium iodate solution of known concentration, then titrating bleach with standardized sodium thiosulfate solution to analyze its properties; this includes finding molarity and percent by mass of sodium hypochlorite in bleach.

\section{Data}

On attached sheet.

\section{Calculations}

On attached sheet.

\section{Graphs}

None.

\section{Conclusion}

From the experiment, we were able to determine the molarity of sodium thiosulfate solution to be 0.0978 M, the molarity of sodium hypochlorite in bleach to be 1.14 M, and the percent by mass of sodium hypochlorite in bleach to be 7.69\%.

\section{Experimental sources of error}

Note: $n$ denotes moles, $M$ denotes molarity, $V$ denotes volume.

\begin{enumerate}
	\item Although enough KI and HCl was added in excess to allow for the full reaction of IO3- in the reaction that generates I2 (shown in the question section), it is still valuable to predict the error if this was not the case. If either KI or HCl were limiting in this reaction, not all of the IO3- ions would be used and we would yield fewer moles of I2 than predicted. This would then result in fewer moles of sodium thiosulfate being needed to reach the equivalence point in the titration than if all the IO3- reacted initially, which corresponds to a smaller volume change of sodium thiosulfate in the buret. Then, since the molarity of sodium thiosulfate is calculated by dividing the moles of sodium thiosulfate by this very volume change $$M = \frac{n}{\Delta V}$$ and the moles of sodium thiosulfate used in the calculation will be higher than the moles used in reality while the volume change of the buret will be smaller, the calculated molarity of sodium thiosulfate will be higher than its true value. This will then have an effect on the analysis of bleach, where we would calculate more moles of sodium thiosulfate used than in reality and thus a greater molarity of NaOCl in bleach as well. This is clear from the equation to calculate the molarity of NaOCl using the sodium thiosulfate concentration, in which the two are directly proportional as follows:
$$M\textsubscript{NaOCl} = \frac{M\textsubscript{Na\textsubscript{2}S\textsubscript{2}O\textsubscript{3}} * \Delta V\textsubscript{Na\textsubscript{2}S\textsubscript{2}O\textsubscript{3}} * \frac{\textrm{1 mol I\textsubscript{2}}}{\textrm{2 mol Na\textsubscript{2}S\textsubscript{2}O\textsubscript{3}}}}{V\textsubscript{bleach}}.$$
We will also see that this makes the percent by mass calculation higher as well, in which the percent by mass and molarity of NaOCl are directly proportional:
$$\textrm{\% mass NaOCl} = \frac{V\textsubscript{bleach} * M\textsubscript{NaOCl} * \frac{\textrm{74.44 g NaOCl}}{\textrm{1 mol NaOCl}}}{m\textsubscript{bleach}} * 100\%.$$
	\item There were two titrations performed in the lab: potassium iodate with sodium thiosulfate (case A) and sodium thiosulfate with bleach (case B). During either of these titrations, if any of the titrant (potassium iodate in A, sodium thiosulfate in B) used did not actually go into the analyte solution (sodium thiosulfate in A, bleach in B), there would be a greater volume change of titrant corresponding to the extra moles required to leave the buret to account for the moles that did not go into solution. There are four cases in which this error could affect the experiment:
	\begin{enumerate}
		\item In titration A, this would result in a smaller calculated molarity of sodium thiosulfate. In the calculation, the moles of sodium thiosulfate is derived entirely from the moles of IO3- and is thus independent of this error while, as mentioned previously, the volume change involved would be greater. Then, because molarity is moles divided by volume, the molarity calculated for sodium thiosulfate would be smaller.
		\item If this same error occurred independently in titration B, the reverse would be true: the calculated molarity of NaOCl would be higher than its actual value. This Is because the molarity calculation in this case involves first finding the moles of sodium thiosulfate used through $$n = M\Delta V.$$
Since the volume change is higher, the moles used would be higher as well. Then, from this, we convert this number into moles of NaOCl used through stoichiometric constants. Overall, we see that the greater moles of NaOCl in the calculation divided by the volume of bleach, which is independent of the volume change of the buret, results in a greater calculated molarity of NaOCl. If the percent by mass calculation was done using this value, its calculated value would also be greater than the actual value, as molarity of NaOCl and percent by mass of NaOCl in bleach are directly proportional as shown previously.
		\item If this error was present in titration A and then the smaller calculated molarity of sodium thiosulfate was used in the calculations for NaOCl, a smaller calculated molarity and percent by mass of NaOCl would result. This is obvious from the direct proportionality of the molarity of sodium thiosulfate and the molarity of NaOCl as previously shown. Using this smaller molarity in the percent by mass calculation would drive it down as well due to their direct proportionality, as previously shown.
		\item In the case where this error is present in both titration A and titration B, it is not possible to determine the overall effect on the molarity and percent by mass calculations for NaOCl without exact data. This is because this error in titration A would drive down the calculated molarity of sodium thiosulfate; in titration B, it would drive up the volume change of sodium thiosulfate in the buret. Since all the bleach calculations rely (proportionally) on the moles of NaOCl and this is derived from $$n\textsubscript{Na\textsubscript{2}S\textsubscript{2}O\textsubscript{3}, used} = M\textsubscript{Na\textsubscript{2}S\textsubscript{2}O\textsubscript{3}} * \Delta V\textsubscript{Na\textsubscript{2}S\textsubscript{2}O\textsubscript{3}}$$ stoichiometrically, it is not possible to say whether the moles of NaOCl will be higher or lower from this error occurring in both titration A and B because this depends on how small the molarity of sodium thiosulfate was found to be in titration A and how large the volume change was in titration B. If the molarity of sodium thiosulfate is disproportionate to the erroneous volume change in titration B, the moles of NaOCl found after stoichiometric manipulations would be smaller and the corresponding conclusions from above would apply. The opposite would be true if the erroneous volume change in titration B was disproportionately high compared to the small molarity found in titration A. To make this more precise, we observe that the moles of sodium thiosulfate calculated can be given by
$$kM_0 * q\Delta V_0 = n\textsubscript{Na\textsubscript{2}S\textsubscript{2}O\textsubscript{3}}, \textrm{where}$$ $$M_0 := \textrm{true molarity of sodium thiosulfate},$$ $$V_0 := \textrm{volume change using true molarity},$$ $$ M := kM_0 \textrm{ and } \Delta V :=  q\Delta V_0.$$
Let us also define $n_0$ to be the true number of moles. All the quantities denoted by subscript $0$ to signify a "true" value are defined in the sense of having this error not affect either titration: they are what you would measure or calculate if 100\% of the titrant entered the solution of titrand. It follows that, in a sense, $k$ and $q$ are multiplicative offsets of the true quantities they correspond to. In the trivial case where $k = q = 1$, $M = M_0$, $\Delta V = \Delta V_0$, and $n = n_0$. If $k = 1/q$, the errors in titrations A and B cancel out and we actually observe the true number of moles of sodium thiosulfate such that $n = n_0$. If $k > 1/q$ then $kq > 1$, and we will calculate a greater number of moles of sodium thiosulfate in the calculation such that $n > n_0$ and observe any effects following from a greater number of moles of sodium thiosulfate as described previously. If $k < 1/q$ then $kq < 1$, we will calculate $n < n_0$ and observe any effects following from a smaller number of moles of sodium thiosulfate as described previously. We can actually see that any case involving both titrations mentioned above can all be represented through this equation by setting $k = 1$ or $q = 1$. Regardless, it was still important to establish the cascading effect of this error for each of the scenarios.
	\end{enumerate}
\end{enumerate}

\section{Questions}
\begin{enumerate}
	\item Calculate the stoichiometric mass of KI and drops of concentrated HCl that must be added to the 25.00 mL pipetted sample of KIO3. Compare with the 1 gram KI and 10 drops of HCl that you added. Were these numbers safely in excess?
	\begin{enumerate}
		\item As shown on the calculations sheet, the minimum stoichiometric mass of KI needed to react completely was 0.415 g while the mass added was 1 g. Thus, the mass of KI added was justifiably in excess because 1 g is almost 2.5x the amount needed.
		\item The minimum stoichiometric moles of HCl needed to react completely was 0.00300 moles, which corresponds to 5 drops as shown on the calculation sheet. Because the 10 drops added was double this minimum amount, it is justifiable to say that we were safely in excess.
	\end{enumerate}
\end{enumerate}

\end{document}
