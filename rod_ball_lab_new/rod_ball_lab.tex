% !TEX TS-program = pdflatex
% !TEX encoding = UTF-8 Unicode

% This is a simple template for a LaTeX document using the "article" class.
% See "book", "report", "letter" for other types of document.

\documentclass[12pt]{article} % use larger type; default would be 10pt

\usepackage[utf8]{inputenc} % set input encoding (not needed with XeLaTeX)

%%% Examples of Article customizations
% These packages are optional, depending whether you want the features they provide.
% See the LaTeX Companion or other references for full information.

%%% PAGE DIMENSIONS
\usepackage{geometry} % to change the page dimensions
\geometry{a4paper} % or letterpaper (US) or a5paper or....
% \geometry{margin=2in} % for example, change the margins to 2 inches all round
% \geometry{landscape} % set up the page for landscape
%   read geometry.pdf for detailed page layout information

\usepackage{graphicx} % support the \includegraphics command and options

% \usepackage[parfill]{parskip} % Activate to begin paragraphs with an empty line rather than an indent

%%% PACKAGES
\usepackage{booktabs} % for much better looking tables
\usepackage{array} % for better arrays (eg matrices) in maths
\usepackage{paralist} % very flexible & customisable lists (eg. enumerate/itemize, etc.)
\usepackage{verbatim} % adds environment for commenting out blocks of text & for better verbatim
\usepackage{subfig} % make it possible to include more than one captioned figure/table in a single float
% These packages are all incorporated in the memoir class to one degree or another...

%%% HEADERS & FOOTERS
\usepackage{fancyhdr} % This should be set AFTER setting up the page geometry
\pagestyle{fancy} % options: empty , plain , fancy
\renewcommand{\headrulewidth}{0pt} % customise the layout...
\lhead{}\chead{}\rhead{}
\lfoot{}\cfoot{\thepage}\rfoot{}

%%% SECTION TITLE APPEARANCE
\usepackage{sectsty}
\allsectionsfont{\sffamily\mdseries\upshape} % (See the fntguide.pdf for font help)
% (This matches ConTeXt defaults)

%%% ToC (table of contents) APPEARANCE
\usepackage[nottoc,notlof,notlot]{tocbibind} % Put the bibliography in the ToC
\usepackage[titles,subfigure]{tocloft} % Alter the style of the Table of Contents
\renewcommand{\cftsecfont}{\rmfamily\mdseries\upshape}
\renewcommand{\cftsecpagefont}{\rmfamily\mdseries\upshape} % No bold!

\usepackage{mathtools}
\usepackage{gensymb}
\usepackage{dcolumn}
\newcolumntype{d}[1]{D{.}{\cdot}{#1}}

%%% END Article customizations

%%% The "real" document content comes below...

\title{Rod Hitting Ball Lab}
\author{Jason Guo}
%\date{} % Activate to display a given date or no date (if empty),
         % otherwise the current date is printed 

\begin{document}
\maketitle

\section{Objective}

To determine the distance the racquetball will land from the point of launch as well as the angle the launcher will rise.

\section{Data}

\begin{center}
    \begin{tabular}{ | l | d{1} | l | }
    \hline
    Quantity & \mathrm{Value} & Unit \\ \hline
    $L$ (rod length) & 38.1 & cm \\ \hline
    $l_{cm}$ (distance from axis to rod-mass COM) & 29.9 & cm \\ \hline
    $l_{top}$ (distance from axis to top of rod) & 2.5 & cm \\ \hline
    $l_{weight}$ (distance from bottom of rod to top of weight) & 2 & cm \\ \hline
    $H$ (final track height above ground) & 98 & cm \\ \hline
    $h_i$ (initial track height) & 9 & cm \\ \hline
    $h_f$ (final track height) & 12.4 & cm \\ \hline
    $R$ (ball radius) & 2.85 & cm \\ \hline
    $M$ (ball mass) & 81.55 & g \\ \hline
    $m$ (total rod \& weight mass) & 101.7 & g \\ \hline
    $m_{rod}$ (rod mass) & 26.54 & g \\ \hline
    $m_{weight}$ (weight mass) & 75.16 & g \\ \hline
    $\theta_i$ (initial rod angle) & 60 & \degree \\ \hline
    $\phi$ (track angle wrt horizontal) & 0.5 & \degree \\ \hline	
    \end{tabular}
\end{center}

\section{Calculations}

\subsection{MOI of rod-mass}

Let $I_{rm}$ be the MOI of the rod-mass about its hinge. The distance $d$ from the axis to the weight's COM $= L - l_{top} - \frac{l_{weight}}{2}$.
$$I_{rm} = \frac{1}{12}m_{rod}L^2 + m_{rod}(\frac{L}{2} - l_{top})^2 + m_{weight}d^2.$$
Thus, $I_{rm} \approx 0.01\, kgm^2$.

\subsection{Angular speed of rod on collision}

From energy conservation, we have
$$mgl_{cm}(1-cos\,\theta_i) = \frac{1}{2}I_{rm}\omega_i^2.$$
Thus, $\omega_i \approx 5.46\, rad/s$.

\subsection{Angular momentum}

Let counterclockwise be positive by convention. We have
$$L_{rm} = I_{rm}\omega_i - Ndt = I_{rm}\omega_f$$
$$L_b = -NRt = -MRv - I_b v/R$$
where $N$ is the normal force between the ball and rod-mass, t is the collision time, $I_b = \frac{2}{5}MR^2$ is the MOI of the ball about its COM, $v$ is the linear velocity of the ball's COM, and $d$ is the same as above.

Here, the angular momentum of the rod-mass $L_{rm}$ is taken wrt the hinge; the angular momentum of the ball $L_b$ is taken wrt its contact point on the track. This eliminates any friction terms that would otherwise be present in our expressions.

If we combine these expressions, we get
$$I_{rm} \omega_i = I_{rm} \omega_f + Mvd + I_b vd/R^2.$$

\subsection{Coefficient of Restitution}

We require another equation to actually solve for $\omega_f$. From a table of CORs, we will estimate that the collision has a COR of $e = 0.7$.
This means that $$e = \frac{v_{tf} - v}{0-v_{ti}} = 0.7$$
where $v_{ti}$ and $v_{tf}$ are the initial and final linear velocities of the rod-mass contact point with the ball, respectively, and $v$ is the ball's translational velocity.
Since the rod-mass contact point is a distance $d$ from the hinge, $v_{ti} = \omega_i d$ and $v_{tf} = \omega_f d$.
Combining, we get that $$v = d(0.7\omega_i + \omega_f).$$

\subsection{Ball velocity (right after collision)}

Substituting the COR expression into angular momentum conservation, we get that $w_f \approx 0.1\,rad/s$ and $v \approx 1.36\,m/s$.

\subsection{Final rod angle}

We reuse the energy conservation expression and find that $\omega_f \approx 0.1\,rad/s$ corresponds to $\theta_f \approx 1\degree$.

\subsection{Ball velocity (before bump)}

By Newton's Second Law,
$$M \frac{dv}{dt} = -F_D - Mgsin\,\phi$$
where $F_D$ is the drag force and $Mgsin\,\phi$ is component of gravity parallel to the track.

We know that $F_D = \frac{1}{2}\rho v^2 C_D A$. The density of air at room temperature $\rho$ is $\approx 1.2\,kg/m^3$, the drag coefficient $C_D$ of the ball is $\approx 0.5$, and the cross-sectional area $A$ of the ball is $\pi R^2$.
Thus, $F_D \approx 0.0008v^2$.

Solving the differential equation with the condition $v(0) = 1.36$, we get that $$v(t) = 2.95tan(.432 - .0290t).$$

We will guess, from the magnitude of velocities present and the track length, that it takes $\approx 2\,s$ to reach the bump; its translational velocity right before the bump $v_{bi} \approx 1.16\,m/s$.

\subsection{Ball velocity (after bump)}

We will neglect drag here because the bump was relatively short.
However, a significant fraction of kinetic energy is transformed into gravitational potential energy.
To simplify calculations, the total kinetic energy $K$ of the ball is $K = K_{linear} + K_{rotation} = \frac{1}{2}Mv^2 + \frac{1}{2}I_b(\frac{v}{R})^2 = \frac{7}{10}Mv^2$ for an arbitrary translational velocity $v$.

By energy conservation,
$$K_i = K_f + Mg(h_f - h_i)$$
since the ball goes up a height of $h_f - h_i$ during the bump.
Solving for the translational velocity right after the bump, we get that $v_{bf} \approx 0.93\,m/s$.

\subsection{Ball velocity (right before projectile motion)}

Now, we will account for drag and the incline once again.
Solving the differential equation again with the new condition $v(0) = 0.93$, we get that $$v(t) = 2.95tan(.305 - .0290t).$$
We make another guess that this portion of the track takes $\approx 3\, s$ to traverse. Finally, we get that the translational velocity right before projectile motion $u \approx 0.65\, m/s$.

\subsection{Ball Travel Distance}

The travel time of the ball $t$ is given by
$$-H = -\frac{1}{2}gt^2 \implies t \approx 0.45\,s.$$
Thus, the ball will travel a distance $ut \approx 0.294\,m = 29.4\,cm$ off the table.

\end{document}
